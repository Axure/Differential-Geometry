\documentclass[11pt]{article}
\usepackage{amsmath}
\usepackage{amsfonts}
\usepackage{amssymb}
\usepackage{amsthm}
\usepackage{graphicx}

\newcommand{\dd}{\mathrm{d}}
\newcommand{\pd}{\partial}
\newcommand{\mr}{\mathbb{R}}

\newtheorem{problem}{Problem}
\numberwithin{problem}{section}
\title{Connection and Curvature}
\author{Hu Zheng \\ Department of Mathematics, Zhejiang University}
\newenvironment{solution}
               {\let\oldqedsymbol=\qedsymbol
                \renewcommand{\qedsymbol}{$\blacktriangleleft$}
                \begin{proof}[\bfseries\upshape Solution:]}
               {\end{proof}
                \renewcommand{\qedsymbol}{\oldqedsymbol}}

\begin{document}

\maketitle
\tableofcontents

\section{Affine Connections}

\begin{problem}

\end{problem}

\section{Riemannian Connections}

\section{Curvature}

\begin{problem}
Prove the second Bianchi identity in the natural frame:
$$R_{ijkl,h}+R_{ijlh,k}+R{ijhk,l}=0$$
\end{problem}

\begin{problem}
Suppose the Riemann curvature tensor $R$ for a Riemannian manifold $(M^m,g)$ satisfy the following:
$$R(X,Y,Z,W)=\dfrac{1}{m-1}\{S(Y,Z)g(X,W)-S(Y,W)g(X,Z)\}$$,
where $S$ is the Ricci tensor, and if $m\ge 3$, $(M^m,g)$ has constant curvature.
\end{problem}

\begin{problem}
Suppose $(M^3,g)$ is a three-dimensional Riemannian manifold. At any $p\in M^3$, take the coordinate system $\{x^i\}$, s.t. at $p$ we have $g_{ij} = \left\langle \dfrac{\pd}{\pd x^i}, \dfrac{\pd}{\pd x^j} \right\rangle=0, i\neq j$. Prove that for $i, j, k\neq$ at $p$, the following holds:
$$R_{ij} = \dfrac{1}{g_kk}R_{ikjk}$$
$$R_{ii}=\dfrac{1}{g_{jj}}R_{ijij}+\dfrac{1}{g_{kk}}R_{ikjk}$$
$$R_{ijij}-g_{ii}R_{jj}-g_{jj}R_{ii}+\dfrac{1}{2}\rho g_{ii} g_{jj}$$
\end{problem}

\begin{problem}
Suppose $(M^m,g)$ is a connected Einstein manifold and $m\ge 3$,

\begin{itemize}
\item[(i)] if $m=3$, then $(M^m,g)$ has constant curvature.

\item[(ii)] if the scalar curvature $\rho$ of $(M^m, g)$ does not vanish, then there is no parallel vector field on $(M^m, g)$.
\end{itemize}
\end{problem}

\begin{problem}
Suppose $M^2\subset \mr^3$ is an immersed surface, and has the Riemannian metric induced by the Euclidean metric on $\mr^3$, prove that the sectional curvature on $M^2$ is the Gauss curvature.
\end{problem}

\begin{problem}
Calculate the sectional curvature, Ricci curvature and scalar curvature of the sphere $$S^m(r) = \left\{x\in \mr ^{m+1} | \sum_i(x^i)^2=r^2 \mathrm{(constant)} \right\}$$. The Riemannian metric on $S^m(r)$ is induced by the Euclidean metric on $\mr^{m+1}$.

\begin{proof}
Let's elaborate on the Riemann curvature tensor.
\end{proof}

\end{problem}

\section{Harmonic Forms}

\begin{problem}
Suppose $\alpha \in A^1(M)$, then $$\int_M(\delta\alpha)\eta = 0$$.
\begin{proof}

\end{proof}

\end{problem}

\begin{problem}
Suppose $M$ is compact smooth Riemannian manifold without boundary, $h,f\in C^2(M)$. Prove the following Green formula:
$$\int_M(h\Delta f-f\Delta h)\eta = 0$$.
\begin{proof}
Recall that $\Delta = -(\dd\delta+\delta\dd)$. $h\Delta$.

According to theorem 3.4.4, we have $(\dd\alpha,\beta)=(\alpha, \delta\beta)$.

we have $$\int_D(h\Delta f)\eta=-\int_D\langle\dd h, \dd f\rangle\eta$$
\end{proof}

\end{problem}

\begin{problem}
Take the geodesic polar coordinates on $(M,g)$:
$$g = (\dd r)^2+g_{ij}(r,\theta)\dd \theta^i\dd \theta^j, 1 \le i,j,k \le m-1$$.
Suppose $f=f(r)$ is independent from $\{\theta^i\}$, prove that
$$\Delta f = f''+\dfrac{m-1}{r}f' + f'\dfrac{\pd}{\pd r}\log \sqrt{G}$$,
where $G=\det(g_{ij})$ and $'$ denotes ordinary differentiation with respect to $r$.
\begin{proof}

\end{proof}

\end{problem}

\end{document}
