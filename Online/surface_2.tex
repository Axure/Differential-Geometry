\documentclass[11pt]{article}
\usepackage{amsmath}
\usepackage{amsfonts}
\usepackage{amssymb}
\usepackage{amsthm}
\usepackage{graphicx}

\newtheorem{problem}{Problem}
\numberwithin{problem}{section}

\newcommand{\dd}{\mathrm{d}}

\newenvironment{solution}
               {\let\oldqedsymbol=\qedsymbol
                \renewcommand{\qedsymbol}{$\blacktriangleleft$}
                \begin{proof}[\bfseries\upshape Solution]}
               {\end{proof}
                \renewcommand{\qedsymbol}{\oldqedsymbol}}

\begin{document}

\section{Surface 2}



\begin{problem}

A surface is either a sphere or plane $\Leftrightarrow H^2 = K$. 

\begin{solution}


Note that $K=\kappa_1 \kappa_2$ and $H = (\kappa_1 + \kappa_2) / 2$. So $H^2=K\Leftrightarrow\\ \kappa_1 = \kappa_2$.

\begin{itemize}

\item $\Rightarrow$


Let's calculate the curvatures for spheres and planes. $?=-\dd \vec{r} \cdot \dd \vec{n}$.
\begin{itemize}

\item For planes we have $\dd \vec{n} = 0$. So everything is identically $0$.

\item For the unit sphere $r = (\cos u \cos v, \sin u \cos v, \sin v )$.
And
$$\vec{r_u}= (-\sin u \cos v, \cos u \cos v, 0), \vec{r_v} = (-\cos u \sin v, -\sin u \sin v, \cos v) $$
\begin{multline}
\vec{n} = \vec{r_u} \times \vec{r_v} / |\vec{r_u}| |\vec{r_v}| \\ =  (\cos u \cos ^2 v, \sin u \cos^2 v, \sin^2 u \cos v \sin v +\sin v \sin u \cos v \cos u)/\cos v \\ = (\cos u \cos v, \sin u \cos v, \sin v)
\end{multline}
We have that $\vec{r} = \vec{n}$. So
\begin{multline}
\dd \vec{r} \\ = (-\sin u \cos v, \cos u \cos v, 0)\dd u + (-\cos u \sin v, -\sin u \sin v, \cos v) \dd v \\ = \dd \vec{n}
\end{multline}
And
\begin{equation}
II = -\dd \vec{r} \cdot \dd \vec{n} = - \cos^2 v du^2 - \sin^2 v dv^2
\end{equation}
\begin{equation}
I = \dd \vec{r} \cdot \dd \vec{r} = -II = \cos^2 v du^2 + \sin^2 v dv^2
\end{equation}
So that the two principle curvatures are all $1$.

\end{itemize}

\item $\Leftarrow$

\end{itemize}







\end{solution}
\end{problem}
\begin{problem}

The helicoid $\vec{r} = (u \cos v, u \sin v, bv) $ is a minimal surface. And that besides planes, all ruled minimal surfaces must be helicoids.

\begin{solution}

Let's recall the definition of minimal surfaces:

Let's calculate their principle curvature.

\end{solution}
\end{problem}
\begin{problem}

If the ?? surface $\vec{r} = (u \cos v, u \sin v, \phi(v))$ is a minimal surface, then it must be the helicoid.

\begin{solution}
Let's calculate their principle curvatures.

\end{solution}
\end{problem}
\begin{problem}

A surface is a minimal surface $\Leftrightarrow$ there exist two families of orthogonal asymptotes.

\begin{solution}


\end{solution}
\end{problem}



\end{document}
